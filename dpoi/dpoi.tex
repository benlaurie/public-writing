\documentclass[a4paper]{article}

\title{Fair Decentralised Consensus Is Impossible}
\author{Ben Laurie \\
ben@links.org}

\begin{document}
\maketitle

\def\mod#1{\,(\textrm{mod}\,#1)}
\def\implies{\Rightarrow}
\def\qe#1{\begin{equation}#1\end{equation}}
\def\qearray#1{\begin{eqnarray}#1\end{eqnarray}}
\def\oneway#1{\textrm{oneway}(#1)}
\def\preoneway#1{\textrm{preoneway}(#1)}

\setlength{\parindent}{0pt}
\setlength{\parskip}{1ex plus 0.5ex minus 0.2ex}

\section{Introduction}

I wrote this proof not because it proves anything that is not obvious,
but because I am tired of descending the cryptocurrency rabbit hole.

If you want to claim that you have a fair decentralised consensus
mechanism, then you have to tell me which of my assumptions is
incorrect for your system. They can't all be correct. I have proof.

Enjoy.

\section{Definitions}

Informally, decentralised means that there is no central
authority. But what does this mean formally? I propose that we can
model a general ``decentralised'' system as a set of participants,
$P$, of unknown size. In other words, no member of $P$ can enumerate
$P$. Also, all members of $P$ are not special in any way.

By decentralised consensus I mean a deterministic algorithm, $C$
which, given a set of possible outcomes (which are also not special),
$O$ and a vote by every member $p$ of $P$ for some outcome $o_p \in
O$, $C(Q) \in O$ where $Q \subseteq P$, and $\exists P' \subset P$
s.t. $C(P') = C(P)$ (in other words, it is possible to determine the
consensus without enumerating $P$).

$C$ is also allowed to fail - i.e. to indicate there is no consensus.

A consensus algorithm $C$ is said to be {\it fair} if  $C(Q)
\in \{o_q : q \in Q\}$ where $Q \subseteq P$ (that is, the consensus
for any subset is voted for by at least one member of that
subset). Note that this is a very weak definition of ``fair'' but is
sufficient for the proof.

\section{Proof}

Consider the point of view of some particular participant, let's say
$q \in Q$ where $Q \subseteq P$.

$q$ must assume\footnote{By which I mean that I can construct $R$ and
  there's no way for $q$ to know that $R$ does not exist.} that
$\exists R \subset P$, $Q \cap R = \emptyset$ (that is, a disjoint
subset of P), $C(Q) \ne C(R)$. This is because of fairness and the
unknowability of $P$: $q$ must assume $R$ exists where all members
have voted for an outcome other than $C(Q)$, which means that $C(R)
\ne C(Q)$, because of fairness. And because no-one is special, $R$
could also meet the consensus rules, whatever they are.

Since no participant is special, $q$ must assume that either $C(Q)$ or
$C(R)$ could be the same as $C(P)$ (note that $C(P)$ could be
neither!), because choosing one would make $q$ special, and hence, $q$
cannot know what $C(P)$ is.

This argument applies to all members of $P$, which implies that no
participant can ever know $C(P)$.

So, in other words, there cannot be fair decentralised consensus.

\section{Afternote}

Actually, there is one: $C$ always fails.

\end{document}
